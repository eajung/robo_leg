\subsection{Mathematical model and simulation of magnetic levitation  spherical  driving  joint  with  inverse  system  decoupling  control \cite{yanneng2012mathematical}}

\textit{The paper presents a new magnetic levitation spherical driving joint. 
It analyses the theory of air-gap magnetic of the driving joint and the mechanism of the generation of the magnetic levitation force and electromagnetic torque which based on the principle of reluctance motor. 
The mathematical model and the inverse system decoupling model of the driving joint have been established. The decoupling linearization and the state feedback closed-loop control system also have been achieved in it. 
The research of the result of system simulation provides us the decoupling characteristic, the dynamic characteristic and capacity of resisting disturbance of the driving joint system.} \\ 
\\
It goes heavily into depth on the mathematical reasoning for the motion equation and inverse system model of magnetic levitation for a spherical driving joint. 
Understanding the mathematical processes behind the motion equation and inverse stabilization will be saved for later on.

\subsection{Modeling  of  a  compliant  joint  in  a  magnetic  levitation  system  for  an  endoscopic camera \cite{tolou2012modeling}}

\textit{A novel compliant Magnetic Levitation System (MLS) for a wired miniature surgical camera robot was designed, modeled and fabricated. 
The robot is composed of two main parts, head and tail, linked by a compliant beam. 
The tail module embeds two magnets for anchoring and manual rough translation. 
The head module incorporates two motorized donut-shaped magnets and a miniaturized vision system at the tip. The compliant MLS can exploit the static external magnetic field to induce a smooth bending of the robotic head (0–80◦), guaranteeing a wide span tilt motion of the point of view.
A nonlinear mathematical model for compliant beam was developed and solved analytically in order to describe and predict the trajectory behaviour of the system for different structural parameters.} \\
\\
This paper breaks down one of the biggest inspirations for the magnetic joint actuators. 
The MLS used has an elastic flexible joint that alters its positioning based from the polarity or strength of the external magnetic field.
It also goes into depth of the mathematical modeling used to understand the motion of this robot.

\newgeometry{margin=2cm} % modify this if you need even more space
\begin{landscape}
% \begin{table}[]
\centering
% \caption{Table of lit review papers that fall under the specific categories.}
% \label{my-label}
% \begin{tabular}{c|lllll|ll|lll|}
% \cline{2-11}
%  & \multicolumn{1}{c}{} & \multicolumn{1}{c}{{\textbf{Actuators}}} & \multicolumn{1}{c}{} & \multicolumn{1}{c}{{\textbf{}}} & \multicolumn{1}{c|}{} & \multicolumn{1}{c}{{\textbf{Type of Exo}}} & \multicolumn{1}{c|}{} & \multicolumn{1}{c}{} & \multicolumn{1}{c}{{ \textbf{Strengths}}} & \multicolumn{1}{c|}{} \\ \cline{2-11} 
%  & \multicolumn{1}{c|}{Pneumatic} & \multicolumn{1}{c|}{Cable} & \multicolumn{1}{c|}{\begin{tabular}[c]{@{}c@{}}Electric Actuated\\ Polymers\end{tabular}} & \multicolumn{1}{c|}{Magnetic} & \multicolumn{1}{c|}{?} & \multicolumn{1}{c|}{Soft} & \multicolumn{1}{c|}{Hard} & \multicolumn{1}{c|}{\textless} & \multicolumn{1}{c|}{Average} & \multicolumn{1}{c|}{\textgreater} \\ \hline
% \multicolumn{1}{|c|}{\textbf{Upper Body}} & \multicolumn{1}{l|}{} & \multicolumn{1}{l|}{} & \multicolumn{1}{l|}{} & \multicolumn{1}{l|}{} &  & \multicolumn{1}{l|}{} &  & \multicolumn{1}{l|}{} & \multicolumn{1}{l|}{} &  \\ \hline
% \multicolumn{1}{|c|}{\textbf{Lower Body}} & \multicolumn{1}{l|}{} & \multicolumn{1}{l|}{} & \multicolumn{1}{l|}{} & \multicolumn{1}{l|}{} &  & \multicolumn{1}{l|}{} &  & \multicolumn{1}{l|}{} & \multicolumn{1}{l|}{} &  \\ \hline
% \multicolumn{1}{|c|}{\textbf{Arm/Hand}} & \multicolumn{1}{l|}{} & \multicolumn{1}{l|}{} & \multicolumn{1}{l|}{} & \multicolumn{1}{l|}{} &  & \multicolumn{1}{l|}{} &  & \multicolumn{1}{l|}{} & \multicolumn{1}{l|}{} &  \\ \hline
% \multicolumn{1}{|c|}{\textbf{Finger}} & \multicolumn{1}{l|}{} & \multicolumn{1}{l|}{} & \multicolumn{1}{l|}{} & \multicolumn{1}{l|}{} &  & \multicolumn{1}{l|}{} &  & \multicolumn{1}{l|}{} & \multicolumn{1}{l|}{} &  \\ \hline
% \multicolumn{1}{|c|}{\textbf{Ankle}} & \multicolumn{1}{l|}{} & \multicolumn{1}{l|}{} & \multicolumn{1}{l|}{} & \multicolumn{1}{l|}{} &  & \multicolumn{1}{l|}{} &  & \multicolumn{1}{l|}{} & \multicolumn{1}{l|}{} &  \\ \hline
% %%%%%%%%%%%%%%
% \end{tabular}
% \end{table}

\begin{table}[]
\centering
\caption{Table to fill out using for literature review.}
\label{my-label}
\begin{tabular}{|c|c|c|c|c|c|c|c|c|c|c|c|c|c|c|}
\hline
\textbf{Paper Name} & \textbf{U.B.} & \textbf{L.B.} & \textbf{Arm/Hand} & \textbf{Finger} & \textbf{Ankle} & \textbf{Pneum.} & \textbf{Cables} & \textbf{Motor} & \textbf{Magnetic} & \textbf{Soft} & \textbf{Hard} & \textbf{\textless} & \textbf{Avg. Str.} & \textbf{\textgreater} \\ \hline
\cite{yanneng2012mathematical} & & & & & & & & & X & & & X & & \\ \hline
\cite{tolou2012modeling} & & & & & & & & & X & & & X & & \\ \hline
% \cite{yanneng2012mathematical} & & & & & & & & & X & & & X & & \\ \hline

\end{tabular}
\end{table}

\end{landscape}
\restoregeometry